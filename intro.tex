\section{Introduction}
\label{intro}

Identify different social foci in one’s friend circle, i.e., the group
of friends of a particular individual. This is similar to community
detection but not entirely the same as community detection. Identifying
social foci is similar to identifying important nodes.

We have used various methods to measure the importance of a node
such as betweenness centrality, closeness centrality and eigenvector
centrality.  We also used k-means clustering to divide the graph into
different clusters and then report centroid of each cluster as foci
for respective cluster.

Our results show that accuracy of betweenness, closeness and eigenvector
centrality results are around \emph{80\%} correct. k-means clustering
results vary with parameters like method of centroid calculation.

%Typical length: 1-1.5 pages.

%Intro text with a citation~\cite{cryptfs}.

%one-page summary of entire paper: - same 4 steps as abstract, but 1
%pgf each, instead of 1 sentence.  - intro very important: most
%reviewers make up their mind early on.
%
%When to write intro: - last: to ensure it properly summarizes entire
%paper.  - first: useful as an outline for rest of work
%
%Often best to write an outline of whole paper/ideas.

The rest of this document is organized as follows.  Section 2
describes the background and motivation.  Section
3 describes our proposed approach.  Section 4 describes evaluation plan
and results.  Section 5 describes related work.  Section 6 describes
how the approach can be extended for other methods,
conclusion and future work.

%%%%%%%%%%%%%%%%%%%%%%%%%%%%%%%%%%%%%%%%%%%%%%%%%%%%%%%%%%%%%%%%%%%%%
%% For Emacs:
% Local variables:
% fill-column: 70
% End:
%%%%%%%%%%%%%%%%%%%%%%%%%%%%%%%%%%%%%%%%%%%%%%%%%%%%%%%%%%%%%%%%%%%%%
%% For Vim:
% vim:textwidth=70
%%%%%%%%%%%%%%%%%%%%%%%%%%%%%%%%%%%%%%%%%%%%%%%%%%%%%%%%%%%%%%%%%%%%%%
% LocalWords:

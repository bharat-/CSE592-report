\section{Introduction}
\label{intro}

An individual's ego network consists of nodes that have direct edge
with the ego, and edges among each other.  There can be different
nodes who are important in spreading of information in different
communities found in the ego network.  And also there are nodes, that
are less important in the network.  Identifying different important
nodes in one’s friend circle, i.e., the group of friends of a
particular individual is similar to community detection but not
entirely the same as community detection.

There are different methods that indicate the importance of a vertex
in a network graph.  In graph theory, indicators of centrality
identify the most important vertices within a graph.  Different
measures of centrality differ how they classify the importance of a
node.  Based on different definitions of importance, there are several
indicators of centrality.
\begin{itemize}
\item
\textbf{Eigenvector Centrality}
A node is important if it is connected to many other important nodes.
The importance of the node is affected by its neighbours.  If we
denote the centrality of vertex $v_i$ by $x_i$, then $$x_{i} =
\frac{1}{\lambda} \sum\limits_{j=1}^n A_{ij}x{j}$$ where $\lambda$ is
a constant, $A$ is the adjacency matrix, $n$ is the number of vertices
in the graph.  Defining the vector of centralities $x = (x_1, x_2,
\cdots)$, we can rewrite this equation in matrix form as $$\lambda x =
A·x$$ and hence we see that $x$ is an eigenvector of the adjacency
matrix with eigenvalue $\lambda$.  Assuming that we wish the
centralities to be non-negative, it can be shown (using the
Perron–Frobenius theorem) that $\lambda$ must be the largest
eigenvalue of the adjacency matrix and $x$ the corresponding
eigenvector~\cite{newman2008mathematics}.
\item
\textbf{Betweenness centrality}
A node is plays important role in spreading any information if there
are multiple paths passing through it.  It denotes how many pairs of
nodes would have to pass through it in order to reach from one to
anther in minimum number of hops~\cite{freeman1977set}.  The
betweenness of a vertex $v_i$ in a graph $G:=(V,E)$ is sum of the
fraction of number of shortest path between any pair of vertices
$(s,t)$ that pass through the vertex $v_i$ to total number of shortest
paths between $(s,t)$ for all pairs of
vertices~\cite{brandes2001faster}.
$$C_B(v)= \sum_{s \neq v \neq t \in
V}\frac{\sigma_{st}(v)}{\sigma_{st}}$$
where $\sigma_{st}$ is total number of shortest paths from node $s$ to
$t$ and $\sigma_{st}(v_i)$ is the number of those paths that pass
through $v_i$.
\item
\textbf{Closeness centrality}
If a node is close to all other nodes, then the nodes will be a
central node for the graph.  Such node can also be important node
since it is in the center of the graph.  The closeness centrality for
node $v_i$ is defined as the reciprocal of the sum of its distances
from all other
nodes~\cite{bavelas1950communication}~\cite{sabidussi1966centrality}.
$$C(v_i)= \frac{1}{\sum_{v_j} d(v_j,v_i)}$$
where $d(v_j,v_i)$ is the shortest path distance in terms of number of
hops between $v_i$ and $v_j$.\\
In our case the maximum distance can be 2 as every node is connected
to ego.  So to remove this limitation we have removed the ego from the
network leaving several different connected networks.  As not all the
nodes are connected now, the distance between two nodes can be
infinity.  In that case the formula described above can not be used.
So for our purpose we have used a modified formula given by Gill and
Schmidt~\cite{gilandschmidt1996}.
$$C(v_i)= \sum_{v_j}\frac{1}{d(v_j,v_i)}$$
This formula takes into account that the two nodes can be disconnected
hence they will contribute a value of zero in the centrality score.
\item
\textbf{K-means Clustering}
All the above mentioned centrality scores gives the important nodes
for a whole ego network.  But there are some nodes who are important
nodes for a particular community or group.  Such nodes should also be
considered important as they are important nodes in their respective
community.  For such node, we first divide the ego network in
different communities using k-means clustering algorithm with user
defined distance algorithm.  Then we report the centroid of each
cluster as important node.  For distance function we use shortest path
distance between two nodes after removing ego from the network.  The
removal of ego from network can leave some nodes as isolated node
since they were connected only to ego.  We consider such nodes part of
no community and never report them as important nodes.
\end{itemize}

The rest of this document is organized as follows.  Section 2
describes experimental setup and results.  Section 3 describes
conclusion and future work.  Section 4 acknowledges the contributors
who were helpful in this project.

%%%%%%%%%%%%%%%%%%%%%%%%%%%%%%%%%%%%%%%%%%%%%%%%%%%%%%%%%%%%%%%%%%%%%
%% For Emacs:
% Local variables:
% fill-column: 70
% End:
%%%%%%%%%%%%%%%%%%%%%%%%%%%%%%%%%%%%%%%%%%%%%%%%%%%%%%%%%%%%%%%%%%%%%
%% For Vim:
% vim:textwidth=70
%%%%%%%%%%%%%%%%%%%%%%%%%%%%%%%%%%%%%%%%%%%%%%%%%%%%%%%%%%%%%%%%%%%%%%
% LocalWords:

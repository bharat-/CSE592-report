\section{Related Work}
\label{related}

%Related work section...
%
%Typical length: 0.5-2.0 papers, avg 1.0 (assuming a 12-14
%conf. paper).
%
%Background and Related Work can be similar.
%Most citations will be in this section.
%
%Describe past work and criticize it, fairly.  Use citations to
%JUSTIFY your criticism! Now you can fairly compare to YOUR work.
%
%Problem: don't put background/motivation material here (too
%late).
%
%Important: when it doubt, cite it! (to avoid not citing a
%reviewer's own papers, or papers they know or "think" are
%related).
%
%If conf. allows unlimited citations, do so.
%
%Organize past work: should it be importance order?
%chronological? categorical?

File based encryption are popular and have been deployed widely.  Matt
Blaze designed Cryptographic File System (CFS) pushes encryption
services into the file system itself~\cite{cfs}.  CFS supports secure
storage at the system level through a standard Unix file system
interface to encrypted files.  Users associate a cryptographic key
with the directories they wish to protect.  Files in these directories
(as well as their pathname components) are transparently encrypted and
decrypted with the specified key without further user intervention;
plain text is never stored on a disk or sent to a remote file server.
CFS can use any available file system for its underlying storage
without modification, including remote file servers such as NFS.
System management functions, such as file backup, work in a normal
manner and without knowledge of the key.

EncFS is a user-space stackable cryptographic file-system similar to
eCryptfs, and aims to secure data with the minimum
hassle~\cite{encfs}.  It uses FUSE to mount an encrypted directory
onto another directory specified by the user.  It does not use a
loopback system like some other comparable systems such as TrueCrypt
and dm-crypt.

Existing cryptographic file systems for Unix do not take into account
that sensitive data must often be shared with other users, but still
kept secret.  By design, the only one who has access to the secret
data is the person who encrypted it and therefore knows the encryption
key or password.  This paper presents a kernel driver for a new
encrypted file system, called Fairly Secure File System (FSFS), which
provides mechanisms for user management and access control for
encrypted files~\cite{fsfs}.  The driver has been specifically
designed with multi user systems in mind.  FSFS also tries to prevent
unintentional transfer of sensitive data to unencrypted file systems,
where it would be stored in plain text.


%%%%%%%%%%%%%%%%%%%%%%%%%%%%%%%%%%%%%%%%%%%%%%%%%%%%%%%%%%%%%%%%%%%%%%%%%%%%%%
%% For Emacs:
% Local variables:
% fill-column: 70
% End:
%%%%%%%%%%%%%%%%%%%%%%%%%%%%%%%%%%%%%%%%%%%%%%%%%%%%%%%%%%%%%%%%%%%%%%%%%%%%%%
%% For Vim:
% vim:textwidth=70
%%%%%%%%%%%%%%%%%%%%%%%%%%%%%%%%%%%%%%%%%%%%%%%%%%%%%%%%%%%%%%%%%%%%%%%%%%%%%%
% LocalWords:  SMR HDDs drive's SMRs

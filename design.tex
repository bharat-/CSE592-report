\section{Design}
\label{Design}

%Opening text...  Typical length: 3-5 pages.
%
%Hardest section to write.  A lot of possible interdependencies.
%
%If you find that you have to have a ``forward'' reference to a
%section of text you've not described yet, it usually means that the
%structure of your paper is wrong.  So avoid fwd refs.  Backward
%references to previous sections is ok, as long as it's not too far in
%the beginning of the paper.
%
%Do an outline of design even print a Table of Contents
%
%Opening: tell reader what to expect.
%
%Open with key design goals, in descending importance.
%
%General rule: whenever you LIST 2 or more items, THINK about their
%order (should it be importance order? chronological?  categorical?)
%
%(A) What are your design goals, and what do they get you?  Separate
%goals with HOW you achieve them.  Possible goals can include:
%
%- improved performance
%
%- improved scalability (same as perf.  but need to test "multiple"
%machines)
%
%- better energy consumption
%
%- improved security (hard to prove "better" security)
%
%- versatility: has more functionality that can be utilized in more
%settings.  A generalization of past specific work.
%
%- compatibility: works with many existing systems, possibly
%unmodified (or with few modifications).
%
%- other design goals?
%
%(B) briefly describe HOW you would accomplish each of your design
%goals.
%
%(C) Show a high-level architectural figure whole system, and describe
%every "box" of section of the figure.
%
%Start with high-level detail of each components, then go into greater
%detail.
%
%(D) Bulk of design: go over every design goal and architectural
%component, and describe it in detail.
%
%Key: don't just say WHAT you did, but WHY you did that.  WHY, WHY,
%WHY!
%
%Tense: past tense for what was designed, present tense to describe
%system operation.  Switch b/t past and present consistently.  NO
%future tense!
%
%-----------------------------------------------------------------------------

\subsection{Centrality}
Various indicators of centrality identify the most important vertices
within a graph.  Most commonly used indicators of centrality are:
\begin{itemize}[leftmargin=*]
\item
Closeness Centrality
\item
Betweenness Centrality
\item
Eigenvector Centrality
\end{itemize}

\paragraph{Closeness Centrality}
Closeness centrality denotes how close a nodes is with different
nodes in the graph.  Central nodes are important, as they can reach
the whole network more quickly than non-central nodes.


\paragraph{Betweenness Centrality}
Node betweenness counts the number of shortest paths that pass through
the node. Betweenness centrality is a measure of how influential a
node is in diffusion of information in the graph.  It is calculated as
fraction of shortest paths that pass through this node.

\paragraph{Eigenvector Centrality}
A nodes is important if it is connected to a number of important
nodes.  Eigenvector centrality denotes the numbers of important nodes
a node is connected to.  Eigenvector centrality corresponds to the
top eigenvector of the adjacency matrix.

\subsection{K-means clustering}
vide the ego network in different clusters using k-means clustering
algorithm.  The centroid of each cluster can be seen as the foci for
respective cluster.  We have used standard MATLAB k-means clustering
APIs to divide the graph into clusters.  We project the data to one
dimension space using spectral clustering technique.


%\subsection{System Operation}
%
%Example of subsection...
%
%%-----------------------------------------------------------------------------
%\paragraph{Garbage collection}
%%
%Example of a pragraph heading.
%
%\begin{figure}[htbp] \begin{centering}
%\epsfig{file=figures/lba-ind-example.eps,angle=270,width=1.00\linewidth}
%\caption{LBA indirection example.  The host first writes LBAs 23,
%352, 53, 63, 64, 65, 52, and 29.  The second write sequence is 75,
%76, 23, 52, 324, 263, and 636.  This causes LBAs 23 and 52 to become
%garbage.  Moreover, reading LBAs sequentially requires reading ABAs
%randomly.} \label{fig:lba-ind} \end{centering} \end{figure}
%
%
%Here's how you refer to Figure~\ref{fig:lba-ind}.
%
%\begin{itemize}
%
%\item itemized list item 1
%
%\item item 2
%
%\end{itemize}
%
%
%\textbf{* TENSE USE: past, present, and future}
%
%By default, everything should be written in PAST tense.  "We designed
%a system and evaluated it."
%
%Use present tense ONLY to describe system operation.  "Our system
%sends a message to the server."
%
%Use future tense ONLY in "future work" section!


%%%%%%%%%%%%%%%%%%%%%%%%%%%%%%%%%%%%%%%%%%%%%%%%%%%%%%%%%%%%%%%%%%%%%%%%%%%%%%
%% For Emacs:
% Local variables:
% fill-column: 70
% End:
%%%%%%%%%%%%%%%%%%%%%%%%%%%%%%%%%%%%%%%%%%%%%%%%%%%%%%%%%%%%%%%%%%%%%%%%%%%%%%
%% For Vim:
% vim:textwidth=70
%%%%%%%%%%%%%%%%%%%%%%%%%%%%%%%%%%%%%%%%%%%%%%%%%%%%%%%%%%%%%%%%%%%%%%%%%%%%%%
% LocalWords:
